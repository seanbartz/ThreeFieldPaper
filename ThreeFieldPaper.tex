\documentstyle[epsfig,12pt]{article}

\topmargin -.8cm
\oddsidemargin  10.5pt
\evensidemargin  10.5pt
\textheight  612pt
\textwidth  432pt

\newcommand{\be}{\begin{equation}}
\newcommand{\ee}{\end{equation}}
\newcommand{\ba}{\begin{eqnarray}}
\newcommand{\ea}{\end{eqnarray}}
\newcommand{\bd}{\begin{displaymath}}
\newcommand{\ed}{\end{displaymath}}

\def\thalf{{\textstyle{\frac{1}{2}}}}
\def\tthalf{{\textstyle{\frac{3}{2}}}}
\def\oneth{{\textstyle{\frac{1}{3}}}}
\def\twoth{{\textstyle{\frac{2}{3}}}}
\def\oneqt{{\textstyle{\frac{1}{4}}}}
\def\ttqt{{\textstyle{\frac{3}{4}}}}
\def\fth{{\textstyle{\frac{4}{3}}}}
\def\ones{{\textstyle{\frac{1}{6}}}}
\def\root{\sqrt{-g}}
\def\Dz{\frac{d}{dz}}
\def\phidot{\dot{\phi}}
\def\phiddot{\ddot{\phi}}
\def\chidot{\dot{\chi}}
\def\chiddot{\ddot{\chi}}
\def\Gdot{\dot{G}}
\def\Gddot{\ddot{G}}
\def\rt6{\sqrt{6}}
\def\mL2{(m_{\phi}L)^2}

\usepackage{graphicx}

\renewcommand{\baselinestretch}{1.0}
\raggedbottom %for right-justified text, remove \raggedright

\title{{\bf A Potential for a Three-Field AdS/QCD Model}}
\author{Sean P. Bartz and Joseph I. Kapusta}

\date{\today}

\parindent=20pt

\begin{document}

\maketitle

\abstract{The Anti-de Sitter Space/Conformal Field Theory (AdS/CFT) correspondence may offer new and useful insights into the non-perturbative regime of strongly coupled gauge theories such as Quantum Chromodynamics (QCD). We present an AdS/CFT-inspired model that describes the spectra of light mesons. The conformal symmetry is broken by a background dilaton field, and chiral symmetry breaking and linear confinement are described by a chiral condensate field. These background fields, along with a background glueball condensate field, are derived from a potential. We describe the construction of the potential, and the calculation of the meson spectra, which match experimental data well. We also argue that the presence of the third background field is necessary to properly describe the meson spectra. The outlook for application of this model to finite temperature systems is also discussed. }

\section{Introduction}

Quantum chromodynamics has been well tested for high-energy collisions, where perturbation theory is applicable. However, at hadronic scales, the interaction is non-perturbative, requiring a new theoretical model. The Anti-de Sitter Space/Conformal Field Theory (AdS/CFT) correspondence establishes a connection between an $n$-dimensional Super-Yang Mills Theory and a weakly-coupled gravitational theory in $n+1$ dimensions. Phenomenological models inspired by this correspondence are known as AdS/QCD, and have succeeded in capturing some features of QCD. 

Quark confinement in QCD sets a scale that is encoded in a cut-off of the fifth dimension in the AdS theory. Soft-wall models use a dilaton as an effective cut-off to limit the penetration of the meson fields into the bulk. The simplest soft-wall models use a quadratic dilaton to recover the linear Regge trajectories, while models that modify the UV behavior of the dilaton more accurately model the ground state masses.

Previous soft-wall models use parametrizations for the background dilaton and chiral fields that are not derived as the solution to any equations of motion. A well-defined action provides a set of background equations from which these fields can be derived. In addition, this action provides access to the thermal properties of the model through perturbation of the geometry.

In this paper, we review previous attempts to find a suitable potential for the background fields of a soft-wall AdS/QCD model. After demonstrating the limitations of models including a dilaton and chiral field alone, we suggest the inclusion of a background glueball field. We then construct a potential that satisfies the necessary UV and IR limits, and use this potential to generate the background fields and calculate the resulting meson spectra.

\section{Review and Motivation}

Previous work showed how to construct a potential for a gravity-dilaton-chiral system, assuming that the fields have power-law behavior, which is accurate in both the UV and IR limits. One of the equations of motion is independent of the choice of potential,
\be
\chidot^2  = \frac{\rt6}{z^2} \Dz(z^2\phidot). 
\label{twofield}
\ee
Examining the IR limit, we know that the dilaton should have quadratic behavior, $\phi(z)=\lambda z^2$. Inserting this into \label{twofield}, we find that the chiral field behaves as
\be
\chi(z)=6^{3/4}\sqrt{\lambda}z,
\ee
which removes one of the independent parameters of the original GKK model. As shown in that paper, the IR coefficient of the chiral field sets the large-$n$ mass splitting between the axial-vector and vector mesons in the model. Using the phenomenological value of $\lambda$, which determines the slope of the Regge trajectories, we find a mass splitting that is much too large.

Because this problem arises in the equation that does not involve the potential, this issue cannot be resolved by the choice of potential in the two-field model. Models that derive the field behavior using the superpotential method suffer from the same problem.

To resolve this problem, we suggest to add an additional scalar field to the model, $G$, representing the glueball condensate. This field must be linear in the IR for linear confinement, and go as $z^4$ in the UV to match the operator dimension in the AdS/CFT dictionary.

\section{Setup}
Consider the action in the string frame for three fields: $\phi$, $\chi$ and $G$ representing the dilaton, a chiral field, and a glueball field with zero mass
\be
S_{string}=\frac{1}{16\pi G_5} \int d^5x \root e^{-2\Phi} \left(R_s\partial_\mu\Phi\partial^\mu\Phi - \thalf\partial_\mu\chi\partial^\mu\chi - \thalf\partial_\mu G \partial^\mu G -e^{-4\Phi/3}V(\phi,\chi,G)\right)
\ee

The potential in the Einstein frame, where the action has its canonical form, is
\be
V(\phi,\chi,G) = {\rm e}^{2\phi/\rt6} \, \tilde{V}(\phi,\chi,G)
\ee
with
\be
\tilde{V} = -12 + 4\sqrt{6}\phi + a_0\phi^2 -\tthalf\chi^2 + \tilde{U}
\label{V}
\ee
Here $\tilde{U}$ is more than quadratic in the fields.  The dilaton mass is undetermined and is not connected to the dimension of the corresponding operator, as discussed by Kapusta and Springer.  It is related to the parameter $a_0$ by $a_0 = \thalf \left[ \mL2-8 \right]$. The potential should be an even function of $\chi$. 

The equations of motion can be written as
\be
\chidot^2 + \Gdot^2 = \frac{\rt6}{z^2} \Dz(z^2\phidot)
\label{C}
\ee
\be
\thalf \rt6 z^2 \phiddot - \tthalf (z\phidot)^2 - 3 \rt6 z\phidot 
-4\sqrt{6}\phi - a_0\phi^2 +\tthalf\chi^2
=  \tilde{U}
\label{U}
\ee
\be
3z\phidot - 2a_0\phi = \frac{\partial \tilde{U}}{\partial \phi}
\label{phi}
\ee
\be
z^2\chiddot -3z\chidot \left(1+\frac{z\phidot}{\rt6} \right) + 3\chi
= \frac{\partial \tilde{U}}{\partial \chi}
\label{chi}
\ee
\be
z^2\Gddot -3z\Gdot \left(1+\frac{z\phidot}{\rt6} \right)
= \frac{\partial \tilde{U}}{\partial G}
\label{G}
\ee


\end{document}